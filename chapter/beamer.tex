\documentclass{beamer}
\usepackage{ctex, hyperref}
\usepackage[T1]{fontenc}

% other packages
\usepackage{latexsym,amsmath,xcolor,multicol,booktabs,calligra}
\usepackage{graphicx,pstricks,listings,stackengine,tikz,array,amssymb,bm}
\usepackage{pgfplots}
\pgfplotsset{compat=1.18}
\usepackage{graphicx}
\usetikzlibrary{shapes,arrows}
\usetikzlibrary{calc}
\usetikzlibrary{positioning, arrows.meta}

\title{社会科学研究方法}
\subtitle{研究方法的哲学基础}
\author{元景阳}
\institute{南京财经大学金融学院(已毕业)}
\date{\today}
\usepackage{kunhuo}

% defs
\def\cmd#1{\texttt{\color{red}\footnotesize $\backslash$#1}}
\def\env#1{\texttt{\color{blue}\footnotesize #1}}
\definecolor{deepblue}{RGB}{0,0,0.5}
\definecolor{deepred}{RGB}{0.6,0,0}
\definecolor{deepgreen}{RGB}{0,0.5,0}
\definecolor{halfgray}{gray}{0.55}

\lstset{
    basicstyle=\ttfamily\small,
    keywordstyle=\bfseries\color{deepblue},
    emphstyle=\ttfamily\color{deepred},    % Custom highlighting style
    stringstyle=\color{deepgreen},
    numbers=left,
    numberstyle=\small\color{halfgray},
    rulesepcolor=\color{red!20!green!20!blue!20},
    frame=shadowbox,
}

\begin{document}
\kaishu

\frame{\titlepage}

\section{存在论}
\begin{frame}
\frametitle{存在论}
\begin{itemize}
\item 世界如一枚硬币,既有可测量的金属材质,又有雕刻其上的图案花纹
\item 想要真正理解世界,在掂量硬币的重量的同时,也要欣赏其花纹与光影
\item "美"是山水的客观属性,还是你内心的主观感受?
\end{itemize}
\end{frame}

\begin{frame}
\frametitle{客观---现实主义}
\begin{itemize}
\item 社会现实是客观存在,独立于个人的意识和认知
\item 社会现象有其独立的客观特征,可以通过科学方法进行观察、测量和分析
\item 强调经验主义或实证主义
\item 认为知识来源于对现实世界的经验观察和数据收集
\item 追求识别社会现象之间存在可识别的因果关系
\end{itemize}
\end{frame}

\begin{frame}
\frametitle{主观---建构主义}
\begin{itemize}
\item 社会现实是通过人类的主观认知和社会互动建构而成
\item 社会现象的意义和现实是由个体和群体共同赋予和解释
\item 强调研究者理解人们如何通过互动和交流来建构社会世界
\item 例如:身份是一种通过个体主观体验及社会互动形成的社会现象
\end{itemize}
\end{frame}

\section{科学革命的开端}

\begin{frame}{亚里士多德的"四因说"}
\begin{enumerate}
\item 质料因(causa materialis):构成事物的物质基础
\item 形式因(causa formalis):事物所具有的结构、形状或本质特征
\item 动力因(causa efficiens):促使事物生成、变化或运动的外在力量
\item 目的因(causa finalis):事物存在或运动的目的与功能
\end{enumerate}
\end{frame}

\begin{frame}
\frametitle{方法论自然主义}
\begin{itemize}
\item 肇始于文艺复兴时期
\item 科学研究的方法论不涉及到宗教思考
\item 超自然存在及终极真理仍然由教会掌握
\item 从仰望神谕转向俯察人间,从依赖经典权威转向依靠观察、实验与理性推理
\end{itemize}
\end{frame}

\begin{frame}
\frametitle{从"为什么"向"如何"的转变}
\begin{itemize}
\item 中世纪学者追问"苹果为何落下"------是否符合上帝的旨意或某种终极目的
\item 科学革命后的思想家们更关注"苹果是如何掉下来的"
\item 试图通过观察和实验揭示支配自然现象的普遍规律
\end{itemize}
\end{frame}

\begin{frame}
\frametitle{唯理主义与经验主义}
\begin{block}{经验主义}
\begin{itemize}
\item 思想渊源可追溯到中世纪晚期的经院哲学家
\item 17至18世纪在英国得到系统发展
\item 强调通过感官经验获得知识
\end{itemize}
\end{block}

\begin{block}{理性主义}
\begin{itemize}
\item 以笛卡尔为代表
\item 强调通过先验推理获得知识
\end{itemize}
\end{block}
\end{frame}

\section{社会思潮的产生与认识论的发展}
\begin{frame}
\frametitle{霍布斯的《利维坦》}
\begin{itemize}
\item 人类因无限欲望陷入持续冲突
\item 为摆脱混乱,需建立拥有绝对权力的国家
\item 提出"社会契约":人们为安全自愿转让权利,形成国家
\item 国家可通过征服或契约建立
\end{itemize}
\end{frame}

\begin{frame}
\frametitle{洛克的革命性改造}
\begin{itemize}
\item 自然状态并非全然"污秽、残酷与短暂"
\item 存在着"自然法"的约束
\item 人拥有生命、自由、财产等不可剥夺的"自然权利"
\item 政府的建立是为了更好地保护这些先在的权利
\item 当政府违背契约、侵犯权利时,人民有权反抗
\end{itemize}
\end{frame}

\begin{frame}
\frametitle{密尔的归纳因果五法}
\begin{enumerate}
\item 求同法:多个不同情境中现象出现的共同因素
\item 求异法:现象出现与不出现的差异因素
\item 求同求异联合法:结合前两种方法
\item 共变法:因素变化与现象变化的关系
\item 剩余法:剔除已知原因后的剩余部分
\end{enumerate}
\end{frame}

\begin{frame}
\frametitle{休谟的因果关系理论}
\begin{itemize}
\item 时间先后:原因必须先于结果
\item 空间邻近:因果事件需在空间上接近
\item 必然联系:基于反复经验形成的心理习惯
\end{itemize}
\end{frame}

\begin{frame}
\frametitle{孔德的实证主义}
\begin{block}{人类理智发展的三个阶段}
\begin{enumerate}
\item 神学的或虚构的阶段
\item 形而上学的或抽象的阶段
\item 实证的或真实的阶段
\end{enumerate}
\end{block}

\begin{block}{科学分类}
数学、天文学、物理学、生物学、社会学
\end{block}
\end{frame}

\begin{frame}
\frametitle{马克思的历史唯物主义}
\begin{itemize}
\item "哲学家们只是用不同的方式解释世界,而问题在于改变世界"
\item 社会规律是人类在物质生产实践中历史性生成的产物
\item 生产力与生产关系的矛盾推动社会革命
\item 无产阶级的历史使命是在革命中改变世界
\end{itemize}
\end{frame}

\section{面向当代的方法论哲学}
\begin{frame}
\frametitle{实用主义}
\begin{itemize}
\item 19世纪70年代由美国哲学家皮尔斯创立
\item 经由詹姆士、杜威等人发展
\item 强调知识和观念的"有用性"
\item "一个观念,只要我们相信它对我们的生活是有益的,那么它就是'真的'"
\end{itemize}
\end{frame}

\begin{frame}
\frametitle{逻辑实证主义}
\begin{block}{四个特点}
\begin{enumerate}
\item 注重"科学语言的逻辑分析"
\item 坚持经验主义认识论
\item 注重归纳方法
\item 强调科学研究与科学方法在获取知识上的逻辑性与客观性的统一
\end{enumerate}
\end{block}
\end{frame}

\begin{frame}
\frametitle{结构主义}
\begin{itemize}
\item 首先兴起于语言学领域
\item 索绪尔的《普通语言学教程》奠定基础
\item 列维-斯特劳斯将其发展并推广到人文学科
\end{itemize}
\end{frame}

\begin{frame}
\frametitle{结构主义方法论的五个原则}
\begin{enumerate}
\item 强调研究对象的整体性
\item 注重系统内部各要素之间的关系
\item 对主体的消解
\item 强调结构的自我调节与自适应
\item 注重结构的动态性与转换
\end{enumerate}
\end{frame}

\begin{frame}
\frametitle{后实证主义}
\begin{block}{四个关键特征}
\begin{enumerate}
\item 注重"经验环境与形而上学的双向互动"
\item 定性与定量相结合的混合研究方法
\item 价值中立原则受到挑战
\item 强调科学知识的"可证伪性"
\end{enumerate}
\end{block}
\end{frame}

\begin{frame}
\frametitle{科学研究的哲学发展路径}
\begin{block}{存在论层面}
\begin{itemize}
\item 客观---现实主义:社会现象具有独立于个体意识的客观规律
\item 主观---建构主义:社会现实由人类认知与互动动态建构
\end{itemize}
\end{block}

\begin{block}{认识论层面}
\begin{itemize}
\item 经验主义:坚持感官经验为知识源头
\item 唯理主义:推崇先验理性与逻辑演绎
\end{itemize}
\end{block}
\end{frame}

\begin{frame}
\frametitle{方法论演进}
\begin{itemize}
\item 从方法论自然主义到实证主义
\item 经验验证与逻辑分析的统一标准
\item 结构主义与后实证主义突破主客观二元对立
\item 强调理论可证伪性、混合研究方法及社会结构的动态性
\end{itemize}
\end{frame}

\begin{frame}
\frametitle{总结}
\begin{itemize}
\item 研究方法的哲学演进反映了社会思潮的发展与分野
\item 现代社会科学在技术化与人文关怀间寻求平衡
\item 既需借助定量工具捕捉规律性
\item 亦需通过定性反思理解意义网络
\item 在实用主义导向的多元方法论融合中回应复杂社会现象
\end{itemize}
\end{frame}

\begin{frame}
\frametitle{参考文献}
\small
\begin{thebibliography}{99}
	\bibitem{1} 
	王正绪,《厘清社会科学研究范式的基本差别》,\href{https://www.cssn.cn/skgz/bwyc/202409/t20240913_5777429.shtml}{中国社会科学网}
	
	\bibitem{2}
	包刚升,\href{https://www.bilibili.com/video/BV1S2HsePERE}{哔哩哔哩:国家起源于人与人的战争状态?如何解读霍布斯的《利维坦》?}
	
	\bibitem{3}
	哈耶克,《自由秩序原理》,三联书店
	
	\bibitem{4}
	Mill, John Stuart条目,\href{https://iep.utm.edu/milljs/}{Internet Encyclopedia of Philosophy}
	
	\bibitem{5}
	Hume, David条目,\href{https://iep.utm.edu/hume/}{Internet Encyclopedia of Philosophy}
	
	\bibitem{6}
	《实证哲学教程》条目,\href{https://www.zgbk.com/ecph/words?SiteID=1&ID=399321}{中国大百科全书网}
	
	\bibitem{7}
	马克思,《关于费尔巴哈的提纲》,\href{https://www.marxists.org/chinese/marx/marxist.org-chinese-marx-1845.htm}{中文马克思主义文库}
	
	\bibitem{8}
	马克思,《<政治经济学批判>序言》,\href{https://www.marxists.org/chinese/marx/06.htm}{中文马克思主义文库}
	
	\bibitem{9}
	杨立华 \ 等,《政治学与公共管理研究方法基础》,北京大学出版社
\end{thebibliography}
\end{frame}

\end{document}